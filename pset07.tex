\documentclass[a4paper]{exam}


\usepackage{amsfonts,amsmath,amsthm}
\usepackage[a4paper]{geometry}
\usepackage{hyperref}
\usepackage{xcolor}

\theoremstyle{definition}
\newtheorem{definition}{Definition}

\newcommand\Z{\ensuremath{\mathbb{Z}}}

\title{Problem Set 07: Proof Methods}
\author{CS/MATH 113 Discrete Mathematics}
\date{Spring 2024}

\boxedpoints

\printanswers

\begin{document}
\maketitle

For each question below, clearly write the statement to prove or disprove so that its logical structure is evident. Provide formal proofs. See \href{https://www.overleaf.com/learn/latex/Theorems_and_proofs#Proofs}{this guide} for typesetting proofs in \LaTeX.

If your proof is exceeding 10 lines, you are probably on the wrong track.

The following definitions may prove helpful when attempting the problems.

\begin{definition}[Prime and composite numbers]
A natural number (0, 1, 2, 3, 4, 5, 6, etc.) is called a \textit{prime number} (or a \textit{prime}) if it is greater than 1 and cannot be written as the product of two smaller natural numbers. The numbers greater than 1 that are not prime are called \textit{composite} numbers.  
\end{definition}

\begin{definition}[Even and odd numbers]
An \textit{even number} is an integer of the form $x=2k$ where $k$ is an integer; an \textit{odd number} is an integer of the form $x=2k+1$.  
\end{definition}

\begin{definition}[Parity]
The \textit{parity of a number} is its property of being even or odd.
\end{definition}

\begin{definition}[Rational number]
A \textit{rational number} can be written as $\frac{p}{q}$ where $p$ and $q$ are integers and $q\neq 0$. A number that is not rational is \textit{irrational}.
\end{definition}

\begin{questions}
  
\question
  Show through contraposition: If $x^2 - 6x + 5$ is even, then $x$ is odd.
  \begin{solution}
    To prove: $Even(x^2 - 6x + 5) \implies Even(x)$, where $x\in\Z$.

    As indicated, we attempt a proof by contraposition.

    \begin{proof}
      Consider $x=2k$ for some $k\in\Z$.\\
      Then, $x^2 - 6x + 5 = 4k^2 - 12k + 5 = 2(2k^2 - 6k + 2) + 1$.
    \end{proof}
  \end{solution}

\question Provide a counterexample to disprove: If $n$ is an integer and $n^2$ is divisible by 4, then $n$ is divisible by 4. Explain why it is a counterexample.

  \begin{solution}
    The given statement is: $n^2\mod 4 = 0 \implies n\mod 4 = 0$, where $n\in\Z$. 

    As indicated, we attempt a disproof by counterexample.

    \begin{proof}
      One counterexample is $n=6$.
    \end{proof}
        
    For this value, $n^2=36=4(9)$ but $n=6$ cannot be written as an integer multiple of 4.
  \end{solution}
  
\question Prove using contradiction that $\sqrt{2}$ is irrational.

  \begin{solution}
    Let $p: \sqrt{2}$ is irrational.

    We prove $p$ by contradiction.

    \begin{proof}
      Assume that $\sqrt{2}$ is rational.\\
      Then $\sqrt{2} =\frac{p}{q}$ for some integers $p$ and $q$ where $q\neq0$ and $\frac{p}{q}$ is irreducible, i.e. it cannot be simplified further.\\
      Then $2 =\frac{p^2}{q^2}$ and $p^2 =2q^2$.\\
      That is, $p^2$ is even, and therefore, using a result from the book, $p$ is also even.\\
      That is, $p=2k_1$ for some $k_1\in\Z$.\\
      Since, $p^2=2q^2$, we get $4k_1^2 = 2q^2$ and $q^2=2k_1^2$.\\
      By similar reasoning as above, $q=2k_2$ for some $k_2\in\Z$.\\
      Then, $\frac{p}{q} = \frac{2k_1}{2k_2}= \frac{k_1}{k_2}$.\\
      But $\frac{p}{q}$ was irreducible. $\qquad {\color{red}\bot}$
    \end{proof}
  \end{solution}
  
\fullwidth{For each of the following problems, clearly mention the proof method that you employ.}

\question Prove that for $n\in\Z$, $n$ is odd if and only if $5n + 6$ is odd.

  \begin{solution}
    To prove: For the domain of $\Z$, $\forall n\exists k_1\exists k_2(n=2k_1+1\iff 5n+5=2k_2+1)$.
    
    This is a biconditional statement and is proved through two conditional statements.

    \begin{proof}[To prove:] if $n$ is odd, then $5n+6$ is odd.\\
      We attempt a direct proof.\\
      We know that $n=2k+1$ from some $k\in\Z$.\\
      Then $5n+6=10k+5+6=10k+11=2(5k+5)+1$.\\
    \end{proof}
    \begin{proof}[To prove:] if $5n+6$ is odd, then $n$ is odd.\\
      We attempt a proof by contraposition.\\
      Assume that $n$ is even. That is, $n=2k$ from some $k\in\Z$.\\
      Then $5n+6=10k+6=2(5k+3)$.
    \end{proof}
    % Enter your solution here.
  \end{solution}

\question Prove or disprove: The sum of a rational and an irrational number is a rational number.

  \begin{solution}
    To prove: The sum of a rational and an irrational number is a rational number.

      We attempt a proof by contradiction.

      \begin{proof}
        Consider a rational number $x =\frac{p_1}{q_1}, q_1\neq 0$ and an irrational number $y$.\\
        Assume that their sum is rational, i.e. $x+y=\frac{p_2}{q_2}, q_2\neq 0$.\\
        Then, $y = x+y - x = \frac{p_2}{q_2} - \frac{p_1}{q_1} = \frac{p_2q_1-p_1q_2}{q_1q_2}$, and $q_1q_2\neq 0$.\\
        That is, $y$ is rational. $\qquad{\color{red}\bot}$
      \end{proof}
    \end{solution}
  
\question Prove or disprove that for $(x^2 - y^2) \mod 4 \neq 2$ where $x$ and $y$ are integers.\\
  \textit{Hint}: a) Consider the different cases of parities of $x$ nd $y$. b) Use the method of \textit{proof by cases} and apply a proof \textit{without loss of generality} described in Section 1.8.2 in the book.

  \begin{solution}
    To prove: $\forall x\in\Z\forall y\in\Z((x^2 - y^2) \mod 4 \neq 2)$.
    
    

    \begin{proof} We consider the possible parities of $x$ and $y$ and prove the statement for each case.
    \begin{proof}[Case 1:] Both $x$ and $y$ are even.\\
      Then $x=2k_1, y=2k_2$ for some integers $k_1$ and $k_2$.\\
      Then $x^2-y^2=4k_1^2-4k_2^2=4(k_1^2-k_2^2)$.\\
      Then $(x^2 - y^2) \mod 4 = 0$.
    \end{proof}
    
    \begin{proof}[Case 2:] Both $x$ and $y$ are odd.\\
      Then $x=2k_1+1, y=2k_2+1$ for some integers $k_1$ and $k_2$.\\
      Then $x^2-y^2=4k_1^2+4k_1+1-4k_2^2-4k_2-1=4(k_1^2+k_1-k_2^2-k_2)$.\\
      Then $(x^2 - y^2) \mod 4 = 0$.
    \end{proof}      
    \begin{proof}[Case 3:] \textit{wlog} $x$ is even and $y$ is odd.\\
      Then $x=2k_1, y=2k_2+1$ for some integers $k_1$ and $k_2$.\\
      Then $x^2-y^2=4k_1^2-4k_2^2-4k_2-1=4(k_1^2-k_2^2-k_2)-1$.\\
      Then $(x^2 - y^2) \mod 4 = 3$.
    \end{proof}
    
    \end{proof}
  \end{solution}

\question 
  Prove or disprove that $2^n + 1$ is prime for every $n\in\Z^+$.

  \begin{solution}
    The given statement is: $2^n + 1$ is prime for every $n\in\Z^+$.

    We provide a disproof by counterexample.

    \begin{proof}
      One counterexample is $n=3$.
    \end{proof}
        
    For this value, $2^n+1=9$ which is not prime.
  \end{solution}
\end{questions}
\end{document}
%%% Local Variables:
%%% mode: latex
%%% TeX-master: t
%%% End: